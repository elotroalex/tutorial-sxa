\setvariables[article][shortauthor={Hudson}, date={July 2017}, issue={2}, DOI={doi:10.7916/D8RJ4WW9}]

\setupinteraction[title={Review of *The Caribbean Memory Project*},author={Peter Hudson}, date={July 2017}, subtitle={Review}]
\environment env_journal


\starttext


\startchapter[title={Review of {\em The Caribbean Memory Project}}
, marking={Review}
, bookmark={Review of *The Caribbean Memory Project*}]


\startlines
{\bf
Peter Hudson
}
\stoplines


If you have spent any time in any archive, you understand that what you find is not always what you are looking for and what you are looking for is not always what you find, and the difficult pleasures of the archive only arrive when one obtains a measure of comfort with this maxim and with the unruly, haphazard, and esoteric organizational regimes constituting most historical repositories. So it was, then, that during a search for one thing, I found something else. A speculative, online query concerning a footnote in the history of Pan-Africanism led to a vast, living repository of the Caribbean: a lazy search for \quotation{George Padmore} opened up, in that inexplicable and surreptitious way, a pathway to Beryl Eugenia McBurnie; and McBurnie opened a portal to one of the smartest, most compelling Caribbean digital projects I have encountered---the Caribbean Memory Project.

\useURL[url1][http://www.caribbeanmemoryproject.com/][][The Caribbean Memory Project]\from[url1] (CMP) was founded in 2014 by Kevin Adonis Browne, a writer and scholar of Caribbean rhetoric based at Syracuse University, and Dawn Cumberbatch, a television and film producer and director based in Port of Spain. Described as \quotation{the Caribbean's first crowd-sourced heritage research platform,} the CMP's organization was spurred by questions concerning the use of digital technologies as a pedagogical, representational, and mnemonic platform to both interpret and articulate Caribbean identities. Its stated aim is \quotation{to activate and engage the memory of cultural heritage among a mixed audience and to aid in counteracting the effects of erasure and forgetting occurring in a growing number of contemporary Caribbean communities.}\footnote{\quotation{The Caribbean Memory Project---About Us,} {\em Caribbean Memory Project}, www.caribbeanmemoryproject.com/about.html.}~

The CMP website is clean, uncluttered, and easy to navigate. Its home page serves as the portal to a number of ancillary projects or subplatforms working within the CMP's mandate. The range is impressive.

Reached from CMP's \quotation{About} page (under the \quotation{Home} tab) is the Mobile Archiving Service (MAS), a fee-based platform for enabling individuals, families, and communitis to document, preserve, and, if so desired, display mementos, historical ephemera, and other archival material. But the CMP offers ways to document stories and images directly on their site for free.

On the homepage, the \quotation{Tell Your Story} tab links to two pages: \quotation{Pass it On!} and \quotation{Tell Me, Nah!} The former invites written testimony on events and themes in Caribbean life and history (a drop-down menu offers options for everything from the 1983 Grenada Revolution to \quotation{Embarrassing Moments}), while the latter leads to an audio archive, allowing visitors to the site to anonymously record ninety-second reminiscences, songs, poems, jokes, and anecdotes about the region (longer recordings produced elsewhere can also be uploaded to the site).

The CMP's \quotation{Research} tab offers links to several archival research projects. \quotation{The Place+Memory Heritage Initiative} is an ambitious crowd-sourced and collaborative cartographic endeavor intended to mark, document, and memorialize the often-unheralded vernacular architectures and local geographies of the Caribbean. Another mapping project, \quotation{The Prior Sketches, 1888,} traces the itinerary of London newspaper illustrator Melton Prior as he traversed the Caribbean, pen and paper in hand. \quotation{The Discarded Archive} displays uploaded images of documents discarded by the National Archive of Trinidad and Tobago, textual refuse that falls outside the mandated collecting policy of that institution but that nevertheless offers particular glimpses into the nation's past, while offering a damning indictment of the archival practices of the present. \quotation{Regarding Peter} uses a letter announcing a family death to explore what Browne describes as the \quotation{vernacular archive} as well as the epistolary imagination of the Caribbean diaspora. \quotation{In Theory: A Notebook Archive} is an intriguing set of handwritten pages tracing the theoretical and philosophical origins and development of Browne's monograph, {\em Tropic Tendencies}. The scanned notebook pages allow the reader a sort of intellectual forensics of the process of research and writing. One wishes more writers and scholars made available to the public such exploratory sketches.

These sites within the greater CMP site largely remain half-built. While the CMP has created a wonderfully open digital architecture---brimming with ideas on Caribbean history, memory, and archive---much of it waits for comments from and the collaboration of the CMP's users. And while these projects offer unlimited potential (and are a real testament to the vision of Browne and Cumberbatch), they do not reflect what is in fact the spine of the CMP project. This can be found through the unassuming, surprisingly modest \quotation{Explore} tab on the home page's banner. Although the \quotation{Explore} tab is not buried in and hidden on the CMP website, one wishes it was bolder and better differentiated from the other links, since it leads to a remarkable living archive of the Caribbean. Clicking on the tab opens a drop-down menu: \quotation{Countries,} \quotation{Documents,} \quotation{People,} \quotation{Organizations,} \quotation{Media,} and \quotation{Your Stories.} At the time of writing, \quotation{Organizations} contained only one link---to the Trinidad Theatre Workshop---and only three items had been uploaded to \quotation{Your Stories.} \quotation{Countries} is divided according to linguistic designation and provides links to countries in the region, each marked by an image of the national flag. Clicking on a given country allows one to browse links to community archives, historical documents, and photographs, as well as to profiles of regional personalities. Again, the section remains very much a work in progress and many of the entries await uploads.

At first glance, the content within the \quotation{People} tab appears to differ little from that of many other digital projects on the Caribbean, with their slightly static capsule descriptions and thumbnail biographies. The Bethyl McBurnie page, to give the example that first brought me to the site, offers a short biography of McBurnie's life and work, drawn from a profile published in {\em National Icons of the Republic of Trinidad and Tobago}, a book issued by the government of Trinidad and Tobago on the occasion of the fiftieth anniversary of independence. Born in 1913, McBurnie was a dancer who left Trinidad for the United States in 1938. She studied at Columbia University under Martha Graham, danced on Broadway, and befriended Paul Robeson before returning to the Caribbean region to organize amateur dance productions while, for the first time in history, putting Pan performances on stage. Her interests in local dance and folkloric practice inspired the work of both Rex Nettleford and Derek Walcott, and her Little Carib Theatre, founded in 1948, with Robeson laying the foundation stone, was the first permanent theater. The productions of McBurnie's Little Carib Dance Company drew from the traditions of {\em j'ouvert} and Pan. McBurnie's work was an instrumental cultural project of the broader political project of postcolonial nation building. McBurnie died in 2000.

This information in the biography is rehearsed, and it is in the information below the entry's main stub that one really gets a sense of McBurnie's importance. There are links, via newspapers.com, to scanned historical newspaper articles about McBurnie. A section of \quotation{found photographs} features dozens of sepia-toned scans of historical images of McBurnie with her friends and colleagues. And beneath the photographs, there is a repository of the present that allows us to move beyond the archive as something static, frozen in the past, and overdetermined by the policies and policing both of the state and of capital. Embedded in the site is a video, from 16 September 2016, of a news report from local broadcaster CCN TV6 about the destruction of McBurnie's Panka Street residence in Port of Spain. It shows a yellow excavator tearing into the rubble of McBurnie's house as a female reporter narrates the scene in voiceover, then cuts to an interview with the owner of the property, who describes, in crass, unsentimental tones, his rationale for its demolition. The preservation of culture or history did not even cross his mind.

The video, an archive documenting the destruction of an archive, forces the users of the CMP into a conversation around memory and preservation practices that goes beyond the mere retelling or recounting of history as if the past is merely past. And this conversation is furthered by the links below the video. At first one sees an Instagram photo by Kevin Adonis Browne labeled \quotation{A Book in Beryl McBurnie's House.} The image depicts amid the rubble a lonely, damaged tome titled, somewhat ironically, {\em Ecology: Man's Relationship to His Environment}. Beneath this image is a link to an impromptu, heartfelt essay on Facebook, written by Trinbagonian artist and activist Rubadiri Victor on the day of the destruction of McBurnie's home. Victor's essay, titled \quotation{The Foundation Stone,} is a lament on the destruction of McBurnie's house and its implications and meanings for the project of Trinidad and Tobago's sovereignty:

\startblockquote
We cannot be serious as a people. Cannot. We must confess now that the whole Independence thing, the flag, the anthem, everything, was just a big joke. A flam ting---and we're ready to move on now to Mc Donald's, the Union Jack, Trump, the Star Spangled, whatever Reality Show boat-ride that comes along. Anything! But this Civilisation called Trinidad and Tobago? We flunking dat exam. Cause we aint serious as big people, as Independent people, as heirs to one of the richest cultural traditions on planet Earth. WE CANNOT BE SERIOUS!!!"
\stopblockquote

For Victor the loss of McBernie's home is one of both patrimony and economic sovereignty:

\startblockquote
The tragedy in Beryl's Home is this. This was a facility that could have been earning millions of dollars. It is one of the top Heritage Sites in African and Folk diaspora dance in the world! A proposal was drafted by the Artists' Coalition of T&T (ACTT) to turn the abandoned house into an Adaptive-Use Heritage Site which would have featured: an interactive Beryl Mc Bernie {[}{\em sic}{]} and Caribbean Dance multi-media Museum; an office for the Beryl Mc Bernie {[}{\em sic}{]} Trust; a working artist studio and small dance studio; and living quarters and working space for Residencies for top international dance practitioners. The Site would have been a pilgrimage point for students and nationals locally and for curious international tourists. Now that dream is no more . . . Part of the charm was the eccentric Beryl-style the building was built and landscaped in . . . The Site was supposed to be listed to be protected by the National Trust!!!\footnote{Rubadiri Victor, \quotation{The Foundation Stone,} {\em Facebook}, 16 September 2016 (capitalization and suspension points in original), \useURL[url2][http://web.archive.org/web/20170904034707/https://www.facebook.com/rubadiri/posts/10157555719545145][][www.facebook.com/rubadiri/posts/10157555719545145]\from[url2].}
\stopblockquote

For Victor, the destruction of the house reflects the myopic vision of the state, a state that had not only abandoned McBurnie in death---except for the most superficial posthumous honors---but had long abandoned her in life as well. His comments are at the heart of what it seems to me the CMP is trying to do: to not merely preserve the past as a series of nostalgic referents for the living, or to document history through a kind of depoliticized {\em Wikipedia}-ized process of mnemonic colonization, but to demonstrate the richness and complexity of the Caribbean past as the living referent and footnote of the present, to make a political argument about regional memory and local history for Caribbean citizens and nations. And indeed, a figure such as McBurnie, although known within dance and performance circles, on one hand, and within the country of her birth, on the other, does not receive the kind of attention of those exiled and diasporic intellectuals, such as C. L. R. James and Stuart Hall, who have the institutional support of the metropolitan centers for the preservation of their memories and the argument about the importance of their work. The CMP, with its focus on local histories and Browne's Caribbean \quotation{vernacular archives,} has transformed Caribbean memory into the febrile and urgent question of Caribbean imagination.

\thinrule

\page
\subsection{Peter Hudson}

Peter James Hudson is the author of {\em Bankers and Empire: How Wall Street Colonized the Caribbean} (2017). He teaches in the departments of history and African American studies at the University of California, Los Angeles.

\stopchapter
\stoptext