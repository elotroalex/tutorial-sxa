\setvariables[article][shortauthor={Sutton}, date={July 2017}, issue={2}, DOI={doi:10.7916/D812651T}]

\setupinteraction[title={The Digital Overhaul of the Archive of *Ecclesiastical and Secular Sources for Slave Societies* (ESSSS)},author={Angela Sutton}, date={July 2017}, subtitle={Digital Overhaul}]
\environment env_journal


\starttext


\startchapter[title={The Digital Overhaul of the Archive of {\em Ecclesiastical and Secular Sources for Slave Societies} (ESSSS)}
, marking={Digital Overhaul}
, bookmark={The Digital Overhaul of the Archive of *Ecclesiastical and Secular Sources for Slave Societies* (ESSSS)}]


\startlines
{\bf
Angela Sutton
}
\stoplines


{\startnarrower\it The archive of the Ecclesiastical and Secular Sources for Slave Societies (ESSSS) digitally preserves endangered ecclesiastical and secular documents related to Africans and Afro-descended peoples in the Americas. It began in 2003 with an NEH grant to explore the potential for gathering digital copies of church records in Cuba and Brazil, and expanded from there. With over 600,000 documents concerning the lives of 4-6 million Africans and their descendants, it is now the largest repository of its kind. The consequences of this rapid yet steady expansion have been an increasingly irrelevant digital platform that is unsustainable and difficult to maneuver. While the archive grows and the information in the ESSSS documents continues to be preserved, fewer researchers are now able to access it. The ESSSS team is currently in the process of updating the archive's platform in order to increase functionality, sustainability, and accessibility. The new archive will be interoperable with the latest digital preservation and dissemination projects, allowing for novel uses of the data. At the heart of this overhaul are several conversations about how to use technology in ways that best centralize underrepresented historic actors, and promote interdisciplinary collaboration between scholars in and of Latin America and the Caribbean. This article discusses the many collaborative decisions between scholars and librarians that went into the digital overhaul of the archive with the hopes of contributing to and exploring the ongoing challenges around combating obsolescence of older digital projects of particular significance to the interconnected histories of Africa and the Americas. \stopnarrower}

\blank[2*line]
\blackrule[width=\textwidth,height=.01pt]
\blank[2*line]

\subsection[reference={background},
bookmark={Background},
title={Background}]

In the fifteenth century, Europeans began to explore West Africa in earnest. As they involved themselves in the slave trades, the Catholic Church mandated the baptism of Africans enslaved by Catholic Europeans. With the introduction of African slavery in the Americas, this mandate extended to the New World's enslaved population.

Once baptized into the church, Africans and their descendants had the option to participate in the sacraments of marriage and Christian burial, securing a place in the historical record. Through membership in the Catholic Church, they also generated a host of other religious records such as confirmations, petitions, and last wills and testaments. In the Iberian American colonies, both enslaved and free Africans joined church brotherhoods (called {\em cofradίas}, {\em cabildos}, or {\em irmandades}) organized along ethnic lines. Through these, Catholic priests recorded not only ceremonial and religious aspects of their lives but also their social, political, and economic networks. Catholic Church records in the Americas therefore became the oldest, longest, and most uniform serial data available for the history of Africans in the Americas.

The church forbids the destruction of these recorded sacraments, so many still exist today in the basements and storage rooms of cathedrals throughout Latin America and the Caribbean, vulnerable to damage or loss from climate, bug infestation, or security concerns. Unfortunately, there are few resources available for preserving these fragile volumes at risk of permanent loss. Most churches do not have the funds or ability to function as archives open to the public, and the dispersed nature of the records in individual churches (rather than centralized archives) also makes them difficult for scholars to access, especially those scholars whose countries can offer little research support. The vast majority of these records have never been, nor ever will be, seen by the public.

\subsection[reference={history-of-essss},
bookmark={History of ESSSS},
title={History of ESSSS}]

In the early 2000s, several scholars independently working in these types of ecclesiastical records to recover the histories of Afro-Latinos applied for outside funding to help make some of these most vital records available. In 2003, the National Endowment for the Humanities recognized the valuable transinstitutional partnership between professors Jane Landers of Vanderbilt University, Mariza Soares de Carvalho of Brazil's Universidade Federal Fluminense, and Paul Lovejoy of the Harriet Tubman Institute for Research on Africa and Its Diaspora at York University in Canada, presenting the team with a grant for \$150,000 to preserve the data in the churches that served Brazil and Cuba's African populations, namely, in Havana, Matanzas, and Niteroi, a suburb of outside of Rio de Janeiro.

\placefigure{Jane Landers, Paul Lovejoy, and Mariza Soares de Carvalho}{\externalfigure[images/JanePaulMariza.jpg]}
At that time, digital preservation projects were still in their infancies, and there were few precedents for this team to draw on. All three professors leaned on their training as historians as well as on whatever time library and IT staff from their institutions could donate to help them figure out a simple system to digitize, store, and curate the images, mainly for personal use. The team built graduate training into their mission from the beginning, taking with them graduate students from all three universities. When they arrived, they found many more volumes than expected, and they trained and hired local graduate students in Cuba and Brazil to help speed the digitization process. Together they photographed each page of the volumes and kept close track of serial numbers and order. The team members returned home with full hard drives; now they were faced with the challenge of how to best manage the data. With no further guidance, the librarians at the time built a database using a locally developed Perl-based framework for presenting the actual images. The ESSSS metadata was stored in a MySQL database and used a Dublin Core-based structure.

At the completion of the project in 2005, more than 120,000 images of rich, underutilized, and at-risk ecclesiastical sources for Africans and persons of African descent in Brazil and Cuba had been preserved and stored at Vanderbilt University, accessible on the university's website. The availability of these records has allowed historians, both academic and public, to re-create the histories of Cuba and Brazil in a more representative fashion, centering on the Africans whose forced labor built these nations.\footnote{See Jane Landers, ed., {\em Slavery and Abolition in the Atlantic World: New Sources and New Findings} (London: Routledge, 2017); Pablo F. Gómez, {\em The Experiential Caribbean: Creating Knowledge and Healing in the Early Modern Atlantic} (Chapel Hill: University of North Carolina Press, 2017); Jane Landers, "New Sources and New Findings for Slavery and Abolition in the Atlantic World," in {\em Slavery and Abolition} 36, no. 5 (2015): 421--23; Jane Landers et al., "Researching the History of Slavery in Colombia and Brazil through Ecclesiastical and Notarial Archives," in Maja Komiko, ed., {\em From Dust to Digital: Ten Years of the Endangered Archives Programme} (Cambridge: Open Book, 2015); David Wheat, {\em Atlantic Africa and the Spanish Caribbean, 1570--1640} (Chapel Hill: University of North Carolina Press, 2016); Solange Mouzinho Alves, "Parentescos e sociabilidades: Experiências familiars dos escravizados no sertão paraibano (São João do Cariri), 1752--1816" (PhD diss., Federal University of Paraíba, 2015); Andréa Ferreira Delgado and Beatriz Gallotti Mamigonian, "Santa Afro Catarina: Digital Collection and Heritage Education," {\em Revista Esboços} 21, no. 31 (2014): 86--108; Andréa Ferreira Delgado and Beatriz Gallotti Mamigonian, "Santa Afro Catarina: Escravidão, espaço e patrimônio na Ilha de Santa Catarina," in A. P. Campos et al., eds., {\em Anais eletrônicos do III Congresso Internacional Ufes / Université Paris-Est / Universidade do Minho: Territórios, poderes, identidades} (Vitória, Brazil: GM Editora, 2011), 1--13; Matheus Silveira Guimarães, "Diáspora africana na paraíba do norte: Trabalho, tráfico e sociabilidades na primeira metade do século XIX" (PhD diss., Federal University of Paraíba, 2015); Lesleyanne Rodrigues de Lima and Sara Kelly de Souza Silva, "Arquivo histórico waldemar bispo duarte e a digitalização de documentos ameaçados," {\em Cadernos Imbondeiro: João Pessoa} 3, no. 2 (2014); Mariza Soares, "Heathens among the Flock: Converting African-Born Slaves in Eighteenth-Century Rio de Janeiro," {\em Slavery and Abolition} 36, no. 3 (2015): 478--94; Renee Soloudre, "Sailing through the Sacraments: Ethnic and Cultural Geographies of a Port and Its Churches---Cartagena de Indias," {\em Slavery and Abolition} 36, no. 3 (2015): 460--77; Flavio dos Santos Gomes, "Slavery and African Identity Patterns in Eighteenth-Century Cuba: Some Directions," {\em Journal of Caribbean History} 44, no. 2 (2010): 224--36; Jane Landers, "Catholic Conspirators? Religious Rebels in Nineteenth-Century Cuba," {\em Slavery and Abolition} 36, no. 3 (2015): 495--20; and Miguel Ramos, "Lucumí (Yoruba) Culture in Cuba: A Reevaluation (1830s--1940s)" (PhD diss., Florida International University, Miami, 2013).} Google Analytics reports show that the ESSSS site receives nearly 50,000 unique visitors monthly; these records have been and are currently being used to produce doctoral dissertations and master's theses at Vanderbilt University, the University of Florida, Florida International University, Harvard University, Notre Dame University, and Michigan State University, in this country; the Universities of São Paulo, Bahia, and Paraíba, in Brazil; and the University of Cartagena, in Colombia.

The success of this project has allowed this group of scholars, and the scholars they have trained, to win further grants from the British Library's Endangered Archives Programme to digitize valuable endangered collections of historic documents throughout Latin America and the Caribbean. In the past fifteen years, these rapidly expanding teams have digitally preserved further key documents from Cuba and Brazil, as well as those in Colombia, Florida (the documents come from the Spanish colonial period), and Cape Verde. In doing so, they have captured an estimated 3 million records concerning more than 7 million enslaved Africans and their descendants as well as 2.5 million non-enslaved persons, many of them free blacks. Also captured in the images is documentation of the Asians, indigenous populations, and Europeans, and their descendants, who worked and lived alongside them through the past four hundred years of history. The project has since expanded beyond the ecclesiastical sources to include notarial material and slave registers. Now the teams, working together with the digital archive of Ecclesiastical and Secular Sources for Slave Societies, have preserved and curated the largest and oldest collection of digitized documents pertaining to Africans and their descendants in the Atlantic world. This information is secure on multiple servers and has backups in multiple versions; it is safe no matter what happens to the hard copies in their original locations.

\placefigure{Cathedral Basilica of St.~Augustine, St.~Augustine, Florida}{\externalfigure[images/Basilica.jpg]}
\subsection[reference={issues-of-expansion},
bookmark={Issues of Expansion},
title={Issues of Expansion}]

As the ESSSS digital archive expanded, it outgrew the initial infrastructure used to manage it. The ESSSS team had in the past, out of necessity, privileged locating and saving endangered information over digital curation. This is because of the specialized and difficult, and often dangerous, nature of rescuing these older documents, which become permanently inaccessible on a regular basis. For example, in the Chocόe region of Colombia, the ESSSS preservation team worked in an area known for being a refuge of the Revolutionary Armed Forces of Columbia. In Cuba, they often labored without consistent access to electricity and running water. One of the collections the ESSSS team digitized went missing from the church where it had been stored, so now the ESSSS photographs are all that remains of the historic record of Africans in the region. Some of the churches, which served plantations in the early modern era, are in locations that are difficult to access or in places vulnerable to climate change: some document collections have been damaged from insect swarms and flooding, for example. Each day these records disappear from churches, and it is a race against time to collect and preserve these early and rare primary sources for African history in the Caribbean and wider Atlantic world.

\placefigure{Insect-Damaged Manuscript}{\externalfigure[images/Insects.jpg]}
The rapid and necessary focus on the preservation mission meant that the images were being kept on an increasingly outdated digital platform that, over time, became unsustainable and difficult to maneuver. The repository grew faster than Vanderbilt's library staff were able to curate it, which resulted in fewer and fewer global researchers able to successfully access or navigate it. This ran contrary to the original mission of ESSSS, which the team shared with the Endangered Archives Programme of the British Library: full and complete open global access to these documents. In particular, the ESSSS team wanted researchers in and of Latin America and the Caribbean to have increased access to their own histories.

\placefigure{Father Bendito with Part of the Archive}{\externalfigure[images/FatherBendito.jpg]}
With recent generous assistance from the American Council of Learned Societies, the Andrew W. Mellon Foundation, the National Endowment for the Humanities, and Vanderbilt University, the ESSSS team is currently in the process of updating the archive's platform to the Sobek Content Management system in order to increase functionality, sustainability, and accessibility. The new archive will be interoperable with the latest digital preservation and dissemination projects, allowing for novel uses of the data.

At the heart of this overhaul are several conversations about how to use technology in ways that best centralize underrepresented historic actors and promote interdisciplinary collaboration between scholars in and of Latin America and the Caribbean. This article discusses the many collaborative decisions between scholars and librarians that went into the digital overhaul of the archive. It hopes to explore and contribute to solutions to the ongoing challenges of obsolescence of older digital projects of particular significance to slavery in the Americas.

\subsection[reference={guerilla-preservation-through-digitization-and-the-goals-of-essss},
bookmark={Guerilla Preservation through Digitization and the Goals of ESSSS},
title={Guerilla Preservation through Digitization and the Goals of ESSSS}]

The endangered nature of ESSSS documents and the ways the document images are collected necessitate a flexible content management system to organize them and make them available to a global public. This is because ESSSS methods of digitization are dependent on seizing available opportunities, capitalizing on existing international relationships, and navigating complex issues of copyright, permissions, and, in the case of Cuba during the US embargo, arranging for ESSSS team members with US passports to gain access to the documents. This made planning difficult, and flexibility became paramount. There was no way to know how many of which kind of documents the ESSSS team could anticipate finding and digitizing. The team learned to become responsive and reactive, since they were not in the position to anticipate and plan for document discoveries.

Of course, funding sources determine what the team can access. The British Library's Endangered Archives Programme is one of the main funders for ESSSS, though other sources are available.\footnote{The team at ESSSS would like to take a moment here to thank all the generous funders who have in the past and who continue to help us fulfil our mission: the British Library's Endangered Archives Programme, the National Endowment for the Humanities, the American Council of Learned Societies, the Andrew W. Mellon Foundation, the Historic St.~Augustine Research Institute, the Diocese of St.~Augustine, the Harriet Tubman Institute, the Laboratório de História Oral e Imagem, and Vanderbilt University.} These sources of funding and their timing often shape how the ESSSS team approaches preservation. Funds may be secured before, during, or, on occasion, after the main work of preservation has been completed.

For the digitization of each potential document collection, the ESSSS team must have some organic connection to people physically near the archive who share the ESSSS goals of historical preservation, international collaboration, and graduate training. Without the collaborators on the ground, gaining access and permissions become nearly impossible. Once team members have identified a potential collection, it is a matter of evaluating candidacy for the ESSSS digital archive. The mission of the archive becomes paramount: Is the information within the collection mainly about Africans and their descendants? Is the information in danger of being lost? Does the collection include the older, more fragile documents from the preindustrial period? Do the documents shed light on not only the enslaved Africans but also the slave society in which they lived?

Then the collection is evaluated for feasibility, namely, can the ESSSS team secure the cooperation of the stewards of the records? The Catholic Church has been very generous, also desiring to see these records preserved and made public in most cases, though each decision ultimately belongs to local priests and bishops. If the ESSSS team can secure access, it must evaluate the facilities and determine logistics and budgets for bringing digitization equipment and the preservation team to the documents. Then the team must find students nearby who could benefit from working in the collection. Each of the ESSSS preservation teams has employed local students to help with digitization efforts. The international ESSSS teams train the students in both digital preservation and history/paleography. In many cases, these students then use these rare documents to write their master's and doctoral dissertations, with their academic fees at least partially funded by the payment they received for hours spent photographing documents for the ESSSS digital archive and/or transcribing them. Without these students' dedication, the ESSSS team would not be able to collect the volumes of images necessary for historians to be able to speak to broader social trends.

Ideally, once each of these conditions has been met, the ESSSS team can proceed with digitization, although sometimes team members are unable to meet all conditions and must then find creative solutions during and sometimes after the digitization process. For example, the ESSSS digital archive does not have permissions to make some of our documents public, for various reasons. While we wait for permissions, these images are inaccessible, though they remain safeguarded in the dark archive until such time that permissions can be secured. On the new site, which will be launched in 2018, users will be able to see that the documents exist and can request to be alerted when they are made public. In certain instances, they may also arrange to visit Vanderbilt University to access them.

It is important to note that this digitization process is not an extraction. All physical documents are left where they were found, though if the project is funded by an EAP grant, ESSSS team members will use the boxes and acid-free paper supplied by the British Library to organize and repackage the documents and other materials, if needed. And because we are aware of the issues that scholars, particularly those from developing nations, have encountered using our outdated site to read the digital images, we always leave behind hard drives full of the digital copies with local archivists, librarians, priests, and students who have indicated interest in receiving these materials. The goal of our project has always primarily been to widen access to the past for everyone.

\placefigure{Digitization Training}{\externalfigure[images/BrazilStudents2015.jpg]}
As each team digitizes, the members collect basic metadata using the Endangered Archives Programme spreadsheet, the program's collection method of choice. These spreadsheets include detailed information related to the organization and curation of the original items as well as their new digital formats.\footnote{Specifically included in the Endangered Archives Programme spreadsheets are the EAP project number, reference number, location of the original material, title of volume, description and scope, provenance/history of the ownership of the volume, system of prior arrangement (if applicable), date, era, extent and format of original material, language, script, physical characteristics of the material, copyright information, and information about the intended digital curation of the documents.} In the past, the team made the decision to rename the files on the camera to reflect the row number in the EAP database table. Team members did this to eliminate overwriting files with the same name from different volumes and projects and to allow the librarians at Vanderbilt to match each image with its metadata. An unforeseeable consequence of this was that as the system grew this naming convention made it difficult to determine the appropriate order of pages in each volume of documents, since some volumes were split into different parts and imported at different times. Fortunately, the ESSSS teams also retained the original file names in the database; in the future these can be catalogued both ways while preserving each image's metadata.

\subsection[reference={content-management},
bookmark={Content Management},
title={Content Management}]

Following the conference "Digital Humanities and the History of Slavery," at Vanderbilt University in 2015, the ESSSS teams moved forward with the plan to switch to a more standardized platform to enable better preservation and dissemination of the unique records and information within.\footnote{Digital Projects represented in this conference included, but are not limited to the Transatlantic Slave Voyages Database, the Digital Library of the Caribbean (dLOC), Baptismal Records Database for Slave Societies (BARDSS), Studies in the History of the African Diaspora Documents (SHADD), Coerced Migration Research Alliance, and Collaborative for Historical Information and Analysis (CHIA).} In the past year, we have worked to transfer the two existing frameworks into a single Microsoft SQL system utilizing SobekCM, an open-source software developed by librarians at the University of Florida, who, along with a larger community of users, support ongoing development and updates. The most well-known collections that use SobekCM are digital libraries such as the University of Florida Digital Collections (UFDC) and the Digital Library of the Caribbean (dLOC). Many other important projects for Caribbean history, including the Historic St.~Augustine Project, the Archive of Haitian Religion and Culture, and the Florida and Puerto Rico Digital Newspaper Project, also use this software. It utilizes C\# programming language to process metadata stored as METS (\useURL[url1][http://web.archive.org/web/20170904034615/http://www.loc.gov/standards/mets/][][www.loc.gov/standards/mets]\from[url1]).

\placefigure{Digital Humanities and the History of Slavery}{\externalfigure[images/Conference.jpg]}
SobekCM organizes the files and treats the volumes as a cohesive unit instead of individual pages. Since the volumes are treated as the base object, exporting the records into MARC XML provides an easy method for the materials to be imported into the library catalogue, where they can be retrieved by WorldCat. It also provides a crosswalk to the Text Encoding Initiative (TEI), the current standard for many new digital humanities projects, particularly in the fields of history and slavery studies. This allows for a wide range of future possibilities when it comes to interoperability of our data with that of larger databases, which scholars have programmed to mine the information in ESSSS and make it more easily searchable. Lastly, files and metadata are stored in directories for each volume and collection. This provides a simple way to share volumes or to do additional analysis of the information within.

Another advantage of using SobekCM to migrate the data is the ability to use a standard XML editor rather than a command line interface to refine and standardize each image's metadata. This allows each member of the ESSSS team the ability to easily work with their images' own metadata and employ the graduate students they train to do the same without any coding skills, rather than separating out this digital task and relegating it to the library staff. Going forward, the team working directly with each set of documents can make decisions concerning metadata that are specific to their collections while adhering to the standards we have determined for the collection as a whole. Each unique collection has its own quirks that can be encoded in the metadata and made searchable. The teams can also choose to either work offline from the location of the documents, creating the metadata while they digitize the images, or create the metadata later.

In June of 2017, Vanderbilt Libraries made the decision to switch over to Fedora operating system, and the ESSSS team followed suit. Fedora offers the same advantages that Sobek CM provided the ESSSS team, with the added benefit of increased institutional support for the archive. Students with limited technical background have also identified Fedora's customizable interface for data entry as easier to utilize than the XML editor we used for Sobek.

\subsection[reference={accessibility},
bookmark={Accessibility},
title={Accessibility}]

The primary efforts of the ESSSS teams to date have been to preserve the invaluable materials being lost daily to neglect, climate, fungi, and other damage. We race against the clock. But we also recognize that there is great demand for better access to the documents we have already digitized, and so we continue to work to address the most pressing issues of accessibility, with the goal of complete open access.

When the ESSSS Digital Archive was first created, the team expected to process only one group of volumes. As a result, the team decided to create a single directory for each of the four versions of the images (tiff, thumbnail, medium-sized jpeg, and full-page jpeg). Upon expansion beyond the initial digitization project, the number of images grew to more than eighty thousand in each directory, which caused a significant delay retrieving the images. This made it difficult to click through each volume on the ESSSS website, particularly from locations with limited bandwidth. As the site grew, the digitized images of the documents became less accessible.

To rectify this problem, now only three versions of the images are created and stored for processing, and these images are optimized, thus requiring less storage. This more standardized platform will allow enhanced growth and dissemination of the ESSSS archive by requiring less bandwidth and giving users multiple options for viewing resolution. Users will also in the future have the ability to download pdfs of each volume to work on offline, rather than having to save each page separately.

All this data, along with the document images and all related source code developed in connection with this transfer, are under the General Public License Version 3.0 (GPLv3) and the Creative Commons Public Domain Declaration (CC0 1.0 Universal), and so may be exported at any time, by any person or institution, through the database website; on completion of the transfer, the code will also go up on Github (\useURL[url2][https://github.com][][github.com]\from[url2]) to facilitate this sharing.

In addition, the ESSSS team instructs graduate students in Spanish and Portuguese paleography using the digitized documents so that the ESSSS website can make their transcriptions of some of the earliest and more unique volumes available, allowing users without such training to access them. It is an ongoing process, and we are experimenting with volunteer transcriptions and OCR software in order to offer more in the future. Other future plans include encoding the transcriptions in the TEI for enhanced searchability, creating English-language translations and accompanying teaching materials for wider use, and ensuring the new website conforms to the guidelines set forth by the Web Access Initiative (WAI).

\subsection[reference={image-curation},
bookmark={Image Curation},
title={Image Curation}]

From the inception, the ESSSS project was created by historians for historians, though interest from a wider public, such as genealogists and archivists as well as students and digital humanists, have affected the ways we organize and present the documents on the website. The website itself will also undergo transition as we move from the OmniUpdate web content management system toward one more suited to our presentation needs.. Planning for the new front face of the digital archive has prompted the ESSSS team to sustain ongoing conversations about intentionality in curation.

The scholars who began the project in 2003 are historians of slavery working in a comparative framework and have therefore approached the scope and design of the landing page within this framework. Accompanying each country's documents are maps, photographs, audiovisual material, and explanations of the type of primary sources included in that country's collection(s), the state in which the source materials were found, and who digitized them. The site also tracks resulting scholarship, latest news about the archive, useful links, and teaching tools for those who wish to use our documents and selected transcriptions in the classroom.

Because the archive is vast and the documents diverse, there are unlimited directions in which to take curation of the volumes. They could be used to tell countless histories about religion and politics and society and economics, or about women, Native American populations, migrant populations, enslaved Africans, resistance, and indentured servitude, to name just a few. The ESSSS team is happy for the documents to be used for each of these purposes, though the context available on the site is arranged purposely for guiding users in how to read these documents to shed light on the history of slavery, the enslaved, and the societies that depended on African slave labor. This is because the team recognizes that the absences of enslaved Africans in particular from the historic record have led to the perpetuating injustices inherent in nations with minority populations of descendants of the enslaved.

The ESSSS team believes the purpose of the archive is to highlight the uniqueness of these records and the ways they can be used to reframe histories of enslavement and race. Each of the countries whose primary sources we digitize still struggles with acknowledging the legacy of slavery and its political, economic, and social consequences. In each, there is great scholarly and popular interest in African history and heritage. Every nation in which ESSSS teams work must respond to this interest in defining national identities in multicultural societies plagued with racial inequality and injustice.

\subsection[reference={moving-forward},
bookmark={Moving Forward},
title={Moving Forward}]

The overhaul of the ESSSS project is an extensive one that requires hundreds of thousands of US dollars and hours of work. We have accomplished a lot in the past year, from identifying the ways our archive was in danger to consulting with some of the best thinkers in the field who have worked on similar projects in order to brainstorm elegant solutions for our overhaul. Our challenges have allowed us to mesh the values of historical preservation, collaboration, and graduate training, on the humanities side, with those of function, sustainability, and accessibility, on the digital side.

Once this conversion and migration is complete, our librarian and digital services coordinator will be able to grow the database organically through processing all new data (such as document images from Cabo Verde, Minas Gerais in Brazil, and other potential new sites, as well as data shared by scholars from around the world) in this more sustainable format as it arrives. This keeps the long-term financial needs of the project to a minimum, yet the option remains for more significant expansion of the database as future funding allows. It also allows other projects to use our raw data in their digital endeavors.

For example, the Baptismal Records Database for Slave Societies (BARDSS) at Michigan State University's Matrix Lab (\useURL[url3][http://web.archive.org/web/20170904034621/http://bardss.matrix.msu.edu/index.html][][bardss.matrix.msu.edu/index.html]\from[url3]) mines ESSSS data and translates our digitized baptismal records into a clean, accessible, and searchable tool that requires no understanding of paleography or Spanish and Portuguese language skills to search. Another such project is the World-Historical Gazetteer of the Collaborative for Historical Information and Analysis (CHIA) at the University of Pittsburgh, which links historical data on a global scale, rendering it available for multiscale global analyses (\useURL[url4][http://www.chia.pitt.edu/][][www.chia.pitt.edu]\from[url4]). The ESSSS team consistently works together with the latest initiatives to help link the data of various digital projects that engage with history of the Atlantic slave trade, while still maintaining and ever-improving its own repository. The information in the ESSSS digital archive allows for a wider and more well-rounded representation of the history of Latin America and the Caribbean in this analysis.

The ESSSS team, in keeping with its mission, is also on a continued quest to address the ongoing work and mission. Team members identify gaps in the scholarship of Africans in the Americas, resulting from the dearth of available primary sources, and address them by searching for suitable document collections to digitize and add to the archive. With the overhaul coming to an end, there will be more resources available to devote to increasing functionality as well as cleaning and enriching the metadata. Most team members work on the ongoing task of cleaning titles and normalizing names to better pinpoint the new search function that will be available shortly, as well as on running a manual checker to protect against bitrot.

Finally, the ESSSS team continually works to secure the future of this archive so that its survival is independent of its Vanderbilt University founder, Professor Landers. We fight future obsolescence by moving beyond disaster recovery toward a sustainable long-term preservation strategy, which includes maintaining membership in the Data Preservation Network (DPN) for some of the more endangered volumes. We also train a wide variety of graduate students and early career scholars to work with the archive in different capacities and understand its history and quirks, to lobby on behalf of it, and to anticipate its ever-evolving needs.

Other planned enhancements include adding new categories of metadata, including the GIS coordinates of documents' initial locations and where they are currently located, the funding sources, and the primary investigators for each volume. The ESSSS digital archive will also in time make it possible for users to add or edit transcriptions online and convert self-contained volumes to pdfs, and it will restrict the already-digitized files (for example, from churches in Angola) that require preservation but for which we do not yet have copyright permission to make them available to the public. We also have the goal of providing researchers a way to create a personal bookshelf to better track the volumes viewed and sort them electronically.

\subsection[reference={conclusion},
bookmark={Conclusion},
title={Conclusion}]

It is easy to look back over a decade and say that the project should have done x or y, but the unknown variables and unanticipated exponential growth of the digital archive meant that no one could have foreseen the very best approach for our repository. The team members did the best they could with what was available to them. In (impossible) hindsight, however, the ESSSS team would have done a few things differently in 2003 in order to save money and facilitate a smoother expansion.

The ESSSS team would have anticipated exponential growth and demand for digitization, allowing more capacity for expansion and uniformity from the inception. Team members also would have anticipated obsolescence and configured the data initially in a way to better fight it. They would have kept better logs and decided on a numbering system independent of the EAP spreadsheets to prevent having to manually reassign image titles later. And while they did reach out to the creators of similar projects, such as the Trans-Atlantic Slave Voyages Database (which in 2003 was in CD ROM format), they would have done more to find other projects and scholars with their own datasets to discuss the possibilities of interoperability and configured data for this from the start.

There was no way to know these things in 2003, but now in 2017, with so many important and vital digital projects going dark, we cannot emphasize enough for digital projects to think big from the inception and anticipate the best possible scenarios. Planning for the unplannable can save a digital project hundreds of thousands of dollars and hours down the line and make a big difference in the longevity of an endeavor like the digital archive of ESSSS.

\thinrule

\subsection[reference={acknowledgments},
bookmark={Acknowledgments},
title={Acknowledgments}]

Thanks to all the international team members, former and current, of the digital archive of Ecclesiastical and Secular Sources for Slave Societies. Although they number too many to thank individually, the archive and this article about its overhaul would have been far poorer without their countless hours of dedicated hard work and collaborative insight. This article could also not have been written without the collaborative environment of the larger community of digital scholars and librarians around the world whose thoughtful conversations and ideas have helped shape the ESSSS digital archive in ways that cannot be fully appreciated in an acknowledgement. A special thank you to Dale Poulter at Vanderbilt Libraries for patiently explaining and fact-checking some of the more technical aspects of this article, though of course any errors here are mine alone.

\thinrule

\page
\subsection{Angela Sutton}

Angela Sutton is a historian of the Atlantic world, with specializations in early modern West Africa, digital history, Caribbean history, and maritime history. She is a postdoctoral fellow at the Digital Humanities Center, developing projects with the digital archive of {\em Ecclesiastical and Secular Sources for Slave Societies} (ESSSS), one of the largest collections of primary sources for Africans in the Americas. She is in the process of revising her dissertation into a manuscript, tentatively titled \quotation{Mercantile Culture of the Gold Coast Slave Trade in the Atlantic World Economy, 1620--1720.} Her archival research in the seventeenth-century English, Dutch, Swedish, and Prussian slave trades has been supported by a number of generous sources, including Vanderbilt University's College of Arts and Sciences, the American Council of Learned Societies (ACLS), the Fulbright Foundation, the Max Kade Center, and Vanderbilt University. In the 2017--18 academic year, she is a CCF research fellow with the John Carter Brown Library.

\stopchapter
\stoptext