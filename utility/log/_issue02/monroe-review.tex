\setvariables[article][shortauthor={Monroe}, date={July 2017}, issue={2}, DOI={doi:10.7916/D8W95NKK}]

\setupinteraction[title={Review of the *Digital Archaeological Archive of Comparative Slavery*},author={J. Cameron Monroe}, date={July 2017}, subtitle={Review}]
\environment env_journal


\starttext


\startchapter[title={Review of the {\em Digital Archaeological Archive of Comparative Slavery}}
, marking={Review}
, bookmark={Review of the *Digital Archaeological Archive of Comparative Slavery*}]


\startlines
{\bf
J. Cameron Monroe
}
\stoplines


\startblockquote
There were no beds given the slaves, unless one coarse blanket be considered such, and none but the men and women had these. . . . They find less difficulty from the want of beds, than from the want of time to sleep; for when their day's work in the field is done, the most of them having their washing, mending, and cooking to do, and having few or none of the ordinary facilities for doing either of these, very many of their sleeping hours are consumed in preparing for the field the coming day. -- Frederick Douglass, {\em The Narrative of the Life of Frederick Douglass}
\stopblockquote

Frederick Douglass's account provides a rare window into the everyday lives of many enslaved Africans in the New World, lives that were consumed by the grueling demands of plantation labor but were also filled with everyday domestic realities.\footnote{Frederick Douglass, {\em The Narrative of the Life of Frederick Douglass} (London: H. G. Collins, 1845), 16--17.} Historical sources are often silent on such aspects of slave life, and those seeking to tease out these hidden histories often engage with a variety of source material. Written accounts, plantation ledgers, slave narratives, oral traditions, music, folklore, and more are standard \quotation{go-to} resources for uncovering the subtle contours of slave life in the past. In the past few decades, archaeological evidence has increasingly entered the conversation, providing a valuable new archive for studying the materiality of slavery in the past. This evidence sometimes corroborates, often counters, and always agitates dominant narratives about slave societies in the New World, and the archaeological study of slavery has expanded significantly since the first excavations were undertaken in slave quarters decades ago.

Numerous publications since the 1980s have brought attention to how the material record of slavery in the New World contributes to our understanding of food ways, domestic life, economy, and cultural values.\footnote{See, for example, Akinwumi Ogundiran and Toyin Falola, eds., {\em Archaeology of Atlantic Africa and the African Diaspora} (Bloomington: Indiana University Press, 2007); and Theresa A. Singleton, ed., {\em \quotation{I, Too, Am America}: Archaeological Studies of African-American Life} (Charlottesville: University Press of Virginia, 1999).} Yet most archaeological projects focus on a single site or, rarely, a handful of sites. As the field has matured, scholars have sought to adopt a comparative approach between sites and across regions, one that is capable of engaging in broader humanistic discourses on slave life and culture in the New World. Finding, accessing, and synthesizing the data necessary for such a comparative approach, however, is often as complicated as digging an archaeological site itself. Artifacts, field notes, and final reports are often squirreled away in either university research labs or, in the best cases, stored in state or local repositories. Additionally, archaeologists often adopt idiosyncratic approaches to mapping, excavation, and artifact analysis. These factors render comparisons between sites a daunting, often impossible task.

Since 2000, with funding from the Andrew W. Mellon Foundation, the National Endowment for the Humanities, and the Thomas Jefferson Foundation, Jillian Galle and Fraser Neiman of the Monticello Department of Archaeology have led an initiative to rectify this issue. Their efforts have resulted in the \useURL[url1][https://www.daacs.org/][][Digital Archaeological Archive of Comparative Slavery]\from[url1], a relational PostgreSQL online database designed to synthesize available archaeological datasets from the Southeast and the Caribbean and to make these data accessible to the public. As the site creators note, the primary goal of DAACS is \quotation{to foster inter-site, comparative archaeological research that will advance our historical understanding of the slavery-based societies that evolved in the Atlantic World during the early-modern era.}\footnote{\quotation{About DAACS} > \quotation{Acknowledgments} > \quotation{The DAACS Research Consortium (2013 to present),} www.daacs.org.} DAACS is becoming increasingly visible in the archaeological community, driving change in the way archaeologists think about teaching and researching slavery in the New World.

DAACS does not just serve as a welcome repository for archaeological datasets created by others, however; it has achieved the hard work of standardizing those same datasets using universal descriptors for artifact types, color data, and stylistic elements, as well as mapping conventions for site maps. In the seventeen years since the project began, the DAACS team has studied collections from eighty-one excavations at thirty-three historic sites, producing a fully query-able database that immeasurably advances our ability to do comparative research. Querying the database yields point-and-click access to data on millions of artifacts, field contexts, objects, and images. The site is of primary value to archaeologists interested in examining data from key archaeological sites and to those interested in wrestling with complex data sets. However, the material presented will also be of serious value to all scholars and students interested in engaging with archaeological evidence. Here one can explore reports, artifact images, and site maps made publicly available, most for the first time, in an accessible online environment.

The DAACS site is organized hierarchically. The \useURL[url2][https://www.daacs.org/][][home page]\from[url2] provides a visually attractive gateway to the site, with a running list of tabs placed along the top, each linking to more specific lines of inquiry. Additionally, a number of prominently displayed features on the home page provide direct links to information of interest to a general audience. One takes you to the \quotation{\useURL[url3][http://web.archive.org/web/20170904035144/https://www.daacs.org/aboutdaacs/][][About DAACS]\from[url3]} page, which provides useful contextual information about the history and goals of the project as well as participants and institutional partners. Another directs the visitor to the \quotation{\useURL[url4][https://www.daacs.org/research/galleries/][][Galleries]\from[url4]} page (under \quotation{Research}), which includes links to a highlights of artifact assemblages from a subset of sites included in the archive. For the novice in archaeology, these are excellent starting points. Here one can find an overview of DAACS history and goals and gain an appreciation for the range of artifacts recovered from plantation sites, before digging deeper into the database.

If your interests rest primarily in gaining a broad, qualitative appreciation for specific sites, the \quotation{\useURL[url5][http://web.archive.org/web/20170904035155/https://www.daacs.org/archaeological-sites-map/][][Archaeological Sites]\from[url5]} tab on the home page will take you to a comprehensive list of excavations included in DAACS. Clicking on any particular excavation, you will be taken to an archaeological map of the site, with links to general information about excavation history and methods, lists of archaeological features, and a discussion of the chronology of features, as well as a comprehensive set of photographs and bibliographic references. This information will be particularly helpful for gaining a general knowledge of a site, locating references to the archaeological contexts, or just getting a sense of the features one might use for further comparative analysis.

If you are interested in digging much deeper into the data, the \quotation{\useURL[url6][http://web.archive.org/web/20170904035202/https://www.daacs.org/query-the-database/][][Query the Database]\from[url6]} tab on the home page will take you to a set of \quotation{queries} that allow you to extract specific information about one site, or a group of sites, located in DAACS. A \useURL[url7][https://www.daacs.org/query-the-database/context-queries/][][context query]\from[url7] will produce information on all of the archaeological contexts and features excavated in the sites selected. In contrast, an \useURL[url8][https://www.daacs.org/query-the-database/artifact-queries/][][artifact query]\from[url8] will yield finds lists for the site(s) selected, allowing you to quantify patterns within a site or compare the relative frequencies of different kinds of artifacts between sites. Such queries, for example, allow you to determine whether enslaved Africans at one site consumed wild foods or utilized locally produced cooking pots to a greater degree than those at another, questions with important implications for our comparative understanding of the economy and culture of slave sites. Furthermore, other queries can be run to locate primary documents about sites, explore site chronology, or retrieve artifact images.

A number of additional features render DAACS an invaluable resource for both research and teaching. First, under the \quotation{\useURL[url9][http://web.archive.org/web/20170904035207/https://www.daacs.org/about-the-database/][][About the Database]\from[url9]} tab are located a series of manuals and guides for cataloguing artifacts, information on how the database was structured, and instructions on interpreting and using DAACS data. Second, under the \quotation{\useURL[url10][https://www.daacs.org/research/][][Research]\from[url10]} tab, the creators have provided a comprehensive bibliography of conference presentations, publications, and theses that have used DAACS data, as well as a bibliography of source material for the sites presented. Third, the \quotation{\useURL[url11][https://www.daacs.org/research/workshops/][][Syllabi and Workshops Handouts]\from[url11]} page (also under the \quotation{Research} tab) contains useful lists of datasets, syllabi, and teaching modules provided by teaching professionals at DAACS and from universities across the country. These are mostly geared toward a university-level audience with some basic skills in quantitative analysis; however, they could be easily modified to suit a more general audience.

In all, DAACS has achieved a monumental task, synthesizing and integrating numerous disparate archaeological data sets into an archive that is both accessible to the general public and useful for scholarly research. Although its primary users are archaeologists, innovative and compelling uses of DAACS data have also come from historians. For example, historians have used DAACS data to document frequent access to firearms by enslaved people in North America and have mined DAACS for evidence of literacy among enslaved populations, in the form of writing slates and slate pencils.\footnote{See Philip D. Morgan and Andrew Jackson O'Shaughnessy, \quotation{Arming Slaves in the American Revolution,} in~Christopher Leslie Brown and Philip D. Morgan, eds., {\em Arming Slaves: From Classical Times to the Modern Age} (New haven: Yale University Press, 2006), 180-208; and Antonio T. Bly, \quotation{Pretends He Can Read}: Runaways and Literacy in Colonial America, 1730--1776,” {\em Early American Studies} 6, no. 2 (2008): 261--94.} DAACS also figures importantly in historians' reflections on the ways archaeological data might advance their understanding of changing slave life ways.\footnote{See Phillip D. Morgan, \quotation{Archaeology and History in the Study of African-Americans,} in Jay B. Haviser and Kevin C. MacDonald, eds., {\em African Re-Genesis: Confronting Social Issues in the Diaspora} (Walnut Creek, CA: Left Coast, 2006), 53--61, and \quotation{The Future of Chesapeake Studies,} in Douglas Bradburn and John C. Coombs, eds., {\em Early Modern Virginia} (Charlottesville: University of Virginia Press, 2011), 300--333.} Thus open source data archives like DAACS offer unique opportunities for interdisciplinary data analysis and dissemination. In DAACS, therefore, there is much grist for the mill for scholars in a wide range of disciplines who seek to gain a more holistic account of the nature of everyday lives of the enslaved in the Southeast and the Caribbean.

\thinrule

\page
\subsection{J. Cameron Monroe}

J. Cameron Monroe is an associate professor of anthropology at the University of California Santa Cruz and the director of the UCSC Archaeological Research Center. He earned a BA from UC Berkeley (1995) and a PhD from UCLA (2003), both degrees in anthropology. Between 2004 and 2006, he was a Postdoctoral Fellow in the departments of African and African American studies, anthropology, and history at Washington University in St.~Louis. He joined the Department of Anthropology at UC Santa Cruz in the fall of 2006. Monroe's research uses an archaeological approach to examine political, economic, and cultural transformation in West Africa and the diaspora in the era of the transatlantic slave trade. Between 2000 and 2013, he conducted field research in the Republic of Benin on the spatial dimensions of political authority and the slave trade in the Kingdom of Dahomey. Since 2015 he has directed research on the materiality of political authority in the Kingdom of Haiti. At UC Santa Cruz, he teaches courses in general archaeology, the archaeology of colonialism, slavery and the slave trade, and spatial analysis.

\stopchapter
\stoptext